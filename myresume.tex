% !TEX program = xelatex

\documentclass{resume}
\usepackage{graphicx}
\usepackage{tabularx}
\usepackage{multirow}
\usepackage{progressbar}
\usepackage{zh_CN-Adobefonts_external} % Simplified Chinese Support using external fonts (./fonts/zh_CN-Adobe/)
%\usepackage{zh_CN-Adobefonts_internal} % Simplified Chinese Support using system fonts

\begin{document}
\pagenumbering{gobble} % suppress displaying page number

%\Large{
%  \begin{tabu}{ c l r }
%   \multirow{5}{1in}{\includegraphics[width=0.88in]{avatar}} & \scshape{Bin Yuan} & {Python~}\progressbar{0.75} \\
%    & \email{yuanbin2014@gmail.com} & {Scala~}\progressbar{0.5} \\
%    & \phone{(+86) 131-221-87xxx} & {Linux~}\progressbar{0.7} \\
%    & \linkedin[billryan8]{https://www.linkedin.com/in/billryan8} & {Flask~}\progressbar{0.5} \\
%    & \github[github.com/billryan]{https://github.com/billryan} & {Javascript~}\progressbar{0.5}
%  \end{tabu}
%}
\Large{
  \begin{tabularx}{\linewidth}{ X r }
    \textbf{\LARGE 樊振豪} & \multirow{4}*{\includegraphics[scale = 0.15]{image_large}} \\
    \email{fanzhenhao@pku.edu.cn} &  \\
    \phone{(+86) 159-1067-0396} &  \\
    \github[github.com/RiceReallyGood]{https://github.com/RiceReallyGood}
  \end{tabularx}
}

\section{\faGraduationCap\ 教育经历}
\datedsubsection{\textbf{北京大学}}{2016.09 -- 至今\quad}
\textit{在读博士研究生}\ 理论物理专业, 预计2021年6月毕业
\datedsubsection{\textbf{浙江大学}}{2012.09 -- 2016.06}
\textit{学士}\ 物理学

\section{\faCogs\ IT 技能}
% increase linespacing [parsep=0.5ex]
\begin{description}[parsep=0.5ex]
  \item[编程语言]: C++, Wolfram Mathematica, \LaTeX, Python, Fortran
  \begin{itemize}
    \item C++: 能够利用C++实现比较大型的数值计算程序
    \item Wolfram Mathematica: 熟练使用Mathematica处理数据
    \item \LaTeX : 熟练使用 \LaTeX 编辑公式,实现论文排版
    \item Python 和 Fortran: 了解常用语法
  \end{itemize}
  \item[平台]: Linux
  \begin{itemize}
    \item 熟悉linux的常用命令,熟悉linux下的项目编译工具make,cmake的使用,
          能够在linux服务器上完成大型的数值计算。
  \end{itemize} 
\end{description}

\section{\faUsers\ 项目经历}
\begin{description}
  \item[\href{https://github.com/RiceReallyGood/Quartic}{费米Hubbard模型的Quartic计算}]: 
    发表\href{https://journals.aps.org/pre/abstract/10.1103/PhysRevE.101.023310}{论文一篇}
  \begin{itemize}
    \item 利用OpenMP实现单个计算节点内多线程并行计算
    \item 利用MPI实现节点间并行
    \item 利用快速傅立叶变换,将原本需要$O(n^2)$的卷积操作,优化到$O(n \log (n))$
    \item 利用Intel 的 MKL 实现了一维费米Hubbard链的精确对角化程序
  \end{itemize} 
  \item[\href{https://github.com/RiceReallyGood/BHM_QMC}{玻色Hubbard模型的Monte Carlo模拟}]\ 
  \begin{itemize}
    \item 用C++ 实现Directed Worm算法
    \item 根据算法中时间片段是排好序的性质,利用二分法,将最频繁的查找操作从$O(n)$
    降低到$O(\log (n))$
  \end{itemize} 
\end{description}
%\datedsubsection{\textbf{FLAG Inc.} California, America}{2012 -- Present}
%\role{Summer Intern}{Manager: xxx}
%Brief introduction: xxx.
%\begin{itemize}
%  \item Implemented xxx feature
%  \item Optimized xxx 5\%
%  \item xxx
%\end{itemize}

%\datedsubsection{\textbf{xxx Projects}}{Jan. 2015 -- Present}
%\role{C, Python, Django, Linux}{Individual Projects, collaborated with xxx}
%Brief introduction: xxx
%\begin{itemize}
%  \item Implemented xxx feature
%  \item Optimized xxx 5\%
%  \item xxx
%\end{itemize}

%\datedsubsection{\textbf{\LaTeX\ résumé template}}{May. 2015 -- Present}
%\role{\LaTeX, Maintainer}{Individual Projects}
%An elegant \LaTeX\ résumé template, https://github.com/billryan/resume
%\begin{itemize}
%  \item Easy to be further customized or extended
%  \item Full support for unicode characters (e.g. CJK) with \XeLaTeX\
%  \item FontAwesome 4.5.0 support
%\end{itemize}

% Reference Test
%\datedsubsection{\textbf{Paper Title\cite{zaharia2012resilient}}}{May. 2015}
%An xxx optimized for xxx\cite{verma2015large}
%\begin{itemize}
%  \item main contribution
%\end{itemize}

% \section{\faCogs\ Skills}
% \begin{itemize}[parsep=0.5ex]
%   \item Programming Languages: C == Python > C++ > Java
%   \item Platform: Linux
%   \item Development: Web, xxx
% \end{itemize}

\section{\faicon{trophy}\ 荣誉和奖励}
\begin{itemize}
  \item \datedline{\textbf{方正奖学金}}{2016-2017}
  \item \datedline{\textbf{北京大学三好学生}}{2016-2017}
  \item \datedline{\textbf{方正奖学金}}{2017-2018}
  \item \datedline{\textbf{优秀科研奖}}{2017-2018}
  \item \datedline{\textbf{方正奖学金}}{2018-2019}
  \item \datedline{\textbf{优秀科研奖}}{2018-2019}
\end{itemize}

%\section{\faInfo\ Miscellaneous}
%\begin{itemize}[parsep=0.5ex]
%  \item Blog: http://your.blog.me
%  \item GitHub: https://github.com/username
%  \item Languages: English - Fluent, Mandarin - Native speaker
%\end{itemize}

%% Reference
%\newpage
%\bibliographystyle{IEEETran}
%\bibliography{mycite}
\end{document}
